\newpage

\section{Actividad 5: Interpretación de las especificaciones del fabricante}

\sangria{} En esta sección se completará el analísis sobre transistores BJT revisando algunos parámetros dados por el fabricante. El objetivo es continuar familiarizandose con los parámetros expresados en los datasheet.

\midTitle{black}{Características eléctricas a $25^oC$}{}

\begin{center}
\begin{tabular}{|
>{\columncolor[HTML]{FFCCC9}}l |l|l|l|l|}
\hline
\cellcolor[HTML]{FFFFC7} & \cellcolor[HTML]{FFFFC7}\textbf{BC548} & \cellcolor[HTML]{FFFFC7}\textbf{BC557} \\ \hline
\textbf{$V_{(BR)CEO}$}   & $30V$                                  &  $-45V$                                \\ \hline
\textbf{$I_{CEO}$}       & $---$                                    &   $---$                                  \\ \hline
\textbf{$I_{CES}$}       &$0.2/15_{max}nA$                        &   $-0.2/-100nA$                        \\ \hline
\textbf{$I_{C}$}         &$100mAdc$                               &     $-100mAdc$                         \\ \hline
\textbf{$I_{EBO}$}       & $---$                                    &                  $---$                   \\ \hline
\textbf{$h_{FE}$}        &$110/800$                               &     $300$                              \\ \hline
\textbf{$h_{fe}$}        &$125/900$                               &         $450/900$                      \\ \hline
\textbf{$V_{BE}$}        &$0.5/0.7V$                              &   $-0.5/-0.7$                          \\ \hline
\textbf{$V_{CE(SAT)}$}   &$0.2V$                                  &      $-0.25/-0.65$                     \\ \hline
\textbf{$P_d$}           &$1.5W$                                  &                $1.5W$                  \\ \hline
\end{tabular}
\end{center}

\begin{center}
\begin{tabular}{|
>{\columncolor[HTML]{FFCCC9}}l |l|l|}
\hline
\cellcolor[HTML]{FFFFC7} & \cellcolor[HTML]{FFFFC7}\textbf{2N2222} & \cellcolor[HTML]{FFFFC7}\textbf{BU208} \\ \hline
\textbf{$V_{(BR)CEO}$}   &   $30V$                                 &         $700V$                               \\ \hline
\textbf{$I_{CEO}$}       &       $---$                               &              $---$                          \\ \hline
\textbf{$I_{CES}$}       &          $---$                            &            $<1mA$                            \\ \hline
\textbf{$I_{C}$}         &       $800mA$                           &              $<5A$                          \\ \hline
\textbf{$I_{EBO}$}       &  $10nA$                                 &               $---$                         \\ \hline
\textbf{$h_{FE}$}        &      $75$                               &               $<2.25$                         \\ \hline
\textbf{$h_{fe}$}        &        $---$                              &           $---$                             \\ \hline
\textbf{$V_{BE}$}        &      $1.3$                              &           $<1.5V$                             \\ \hline
\textbf{$V_{CE(SAT)}$}   &        $400mV$                          &             $<5V$                           \\ \hline
\textbf{$P_d$}           &        $500mW$                          &           $1.25W$                             \\ \hline
\end{tabular}
\end{center}

\begin{center}
\begin{tabular}{|
>{\columncolor[HTML]{FFCCC9}}l |l|l|}
\hline
\cellcolor[HTML]{FFFFC7} & \cellcolor[HTML]{FFFFC7}\textbf{MPS6514} & \cellcolor[HTML]{FFFFC7}\textbf{TIP36} \\ \hline
\textbf{$V_{(BR)CEO}$}   &          $25V$                           &        $40V$                                \\ \hline
\textbf{$I_{CEO}$}       &              $---$                         &        $1mA$                              \\ \hline
\textbf{$I_{CES}$}       &                 $---$                      &         $0.7mA$                               \\ \hline
\textbf{$I_{C}$}         &             $200mA$                      &            $25A$                            \\ \hline
\textbf{$I_{EBO}$}       &                         $---$              &           $40V$                             \\ \hline
\textbf{$h_{FE}$}        &   $150/300$                              &        $10/100$                                \\ \hline
\textbf{$h_{fe}$}        &         $---$                              &              $25$                          \\ \hline
\textbf{$V_{BE}$}        &         $---$                              &                  $2V$                      \\ \hline
\textbf{$V_{CE(SAT)}$}   &       $0.5V$                             &              $4V$                          \\ \hline
\textbf{$P_d$}           &  $625mW$                                 &           $125W$                             \\ \hline
\end{tabular}
\end{center}

\newpage{}

\midTitle{black}{Características térmicas}{}
\begin{center}
\begin{tabular}{|l|l|l|l|l|}
\hline
\rowcolor[HTML]{FFFFC7} 
                                                & \textbf{BC548} & \textbf{BC557} & \textbf{2N2222} & \textbf{BU208} \\ \hline
\cellcolor[HTML]{FFCCC9}\textbf{$\theta _{jc}$} &   $83.3°C/W$   &  $83.3°C/W$    &    $146K/W$     &    $1.6°C/W$          \\ \hline
\cellcolor[HTML]{FFCCC9}\textbf{$\theta _{ja}$} &   $200°C/W$    &  $200°C/W$     &   $350K/W$      &       $---$         \\ \hline
\end{tabular}
\end{center}
\begin{center}
\begin{tabular}{|l|l|l|}
\hline
\rowcolor[HTML]{FFFFC7} 
                                                & \textbf{MPS6514} & \textbf{TIP36} \\ \hline
\cellcolor[HTML]{FFCCC9}\textbf{$\theta _{jc}$} &   $83.3°C/W$     &      $1°C/W$          \\ \hline
\cellcolor[HTML]{FFCCC9}\textbf{$\theta _{ja}$} &   $200°C/W$      &       $$---$$                \\ \hline
\end{tabular}
\end{center}


\midTitle{black}{Características conmutación a $25^oC$}{}

\begin{center}
\begin{tabular}{|l|l|l|l|l|}
\hline
\rowcolor[HTML]{FFFFC7} 
                                           & \textbf{BC548} & \textbf{BC557} & \textbf{2N2222} & \textbf{BU208} \\ \hline
\cellcolor[HTML]{FFCCC9}\textbf{$t _{on}$} &       $---$      &      $---$       &       $35nS$    &$---$\\ \hline
\end{tabular}
\end{center}
\begin{center}
\begin{tabular}{|l|l|l|}
\hline
\rowcolor[HTML]{FFFFC7} 
                                          & \textbf{MPS6514} & \textbf{TIP36} \\ \hline
\cellcolor[HTML]{FFCCC9}\textbf{$t_{on}$} &        $---$          &      $---$          \\ \hline
\end{tabular}
\end{center}


\midTitle{black}{Significados}{}

\textbf{$V_{(BR)CEO}$} $\rightarrow$ Tensión de ruptura colector-emisor con la base abierta. Es el voltaje máximo que puede aplicarse entre colector y emisor sin que el transistor conduzca de forma descontrolada cuando la base está desconectada. \\

\textbf{$I_{CEO}$} $\rightarrow$ Corriente de fuga colector-emisor con base abierta. Es la pequeña corriente que fluye del colector al emisor cuando el transistor está en corte. \\

\textbf{$I_{CES}$} $\rightarrow$ Corriente de fuga colector-emisor con base unida a emisor. Indica la corriente que fluye del colector al emisor con la base y el emisor cortocircuitados. \\

\textbf{$I_{C}$} $\rightarrow$ Corriente de colector. Es la corriente principal que circula desde el colector al emisor cuando el transistor está en conducción. \\

\textbf{$I_{EBO}$} $\rightarrow$ Corriente de fuga emisor-base con colector abierto. Representa la corriente inversa mínima entre emisor y base cuando el colector está desconectado. \\

\textbf{$h_{FE}$} $\rightarrow$ Ganancia de corriente continua del transistor (también conocida como $\beta$). Relación entre la corriente de colector y la de base en modo activo. \\

\textbf{$h_{hf}$} $\rightarrow$ Ganancia de corriente en alta frecuencia. Es el parámetro de ganancia de corriente en respuesta a señales de frecuencia más alta. \\

\textbf{$V_{BE}$} $\rightarrow$ Tensión base-emisor. Es el voltaje necesario entre la base y el emisor para que el transistor comience a conducir (típicamente 0.6–0.7V en transistores de silicio). \\

\textbf{$V_{CE(SAT)}$} $\rightarrow$ Tensión colector-emisor en saturación. Es el voltaje entre colector y emisor cuando el transistor está totalmente encendido (saturado). Suele ser bajo (por ejemplo, 0.2V). \\

\textbf{$P_d$} $\rightarrow$ Potencia máxima de disipación. Indica la máxima potencia que el transistor puede disipar como calor sin dañarse. \\

\textbf{$\theta_{ja}$} $\rightarrow$ Resistencia térmica unión-ambiente. Mide cuán eficientemente el transistor transfiere calor desde la unión interna al aire circundante. \\

\textbf{$\theta_{jc}$} $\rightarrow$ Resistencia térmica unión-carcasa. Mide cuán eficientemente el calor pasa de la unión del transistor a su encapsulado. \\

\textbf{$t_{on}$} $\rightarrow$ Tiempo de encendido. Es el tiempo que tarda el transistor en pasar del estado de corte al de saturación. \\