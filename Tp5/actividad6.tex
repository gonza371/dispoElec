\section{Actividad 6: Control de disparo del TRIAC}

\subsection{Actividad de Laboratorio}

\begin{itemize}
    \item Circuito Dimmer.
    \item Transformador de aislación.
    \item Multimetro, osciloscopio analógico y punta de medición.
\end{itemize}

\paragraph{Procedimiento}
Se montó el siguiente circuito:\\
\includegraphics[width=8cm]{./imagenes/Circ6.png}

Conectamos el transformador de aislación a la red eléctrica y luego conectamos el osciloscopio a la salida del transformador.\\
Llevamos el potenciómetro hacia el valor óhmico más alto, observamos y graficamos la forma de onda observada en el osciloscopio.\\
Ahora variamos el valor del potenciómetro mientras observamos la forma de onda en el osciloscopio, graficando las diferentes formas de onda obtenidas para distintos valores del potenciómetro.\\
Luego medimos con el osciloscopio el valor de $I_H$ que produce el apagado del TRIAC al final de cada semiciclo.\\