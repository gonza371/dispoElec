\newpage

\section{Actividad 7: Interpretación del Datasheet}

\subsection{Parametros del DIAC}

\begin{itemize}
    \item \textbf{VBO} (Voltaje de ruptura): Es el voltaje al cual el DIAC comienza a conducir corriente en ambas direcciones. En este caso, es de 28V el mínimo, 32V el típico y 36V el máximo.
    \item \textbf{IBO} (Corriente de ruptura): Es la corriente que fluye a través del DIAC cuando alcanza el voltaje de ruptura. En este caso, es de 50µA.
    \item \textbf{$\Delta$V} (Diferencia de voltaje): Es la diferencia de voltaje entre el punto de ruptura y el voltaje de alimentación. En este caso, es de 5V.
    \item \textbf{IR} (Corriente filtrada): Es la corriente que se filtra por el dispositivo tanto en directa como inversa. En este caso, es de 10µA.
    \item \textbf{IP} (Corriente pico): Es la corriente máxima que puede soportar el DIAC en un pulso corto. En este caso, es de 0.3A.
\end{itemize}

\subsection{Parametros del SCR}

\begin{itemize}
    \item \textbf{VDRM} (Voltaje máximo directo de bloqueo): Es el voltaje máximo que el SCR puede soportar en polarización directa sin dispararse. En este caso, es de 600V.
    \item \textbf{IT(RMS)} (Corriente RMS máxima): Es la corriente máxima que el SCR puede manejar en condiciones normales de operación. En este caso, es de 4A.
    \item \textbf{IT(AV)} (Corriente media máxima): Es la corriente media máxima que el SCR puede manejar en un ciclo completo. En este caso, es de 2.55A.
    \item \textbf{ITSM} (Corriente de pulso máxima): Es la corriente máxima que el SCR puede manejar en un pulso corto. En este caso, es de 20A.
    \item \textbf{IDRM} (Corriente de fuga en bloqueo directo): Es la corriente que fluye a través del SCR cuando está en polarización directa pero no está disparado. En este caso, es de 10µA.
    \item \textbf{IGT} (Corriente de disparo): Es la corriente mínima que se debe aplicar a la compuerta para disparar el SCR. En este caso, es de 15µA la típica y 200µA la máxima.
    \item \textbf{VGT} (Voltaje de disparo): Es el voltaje mínimo que se debe aplicar a la compuerta para disparar el SCR. En este caso, es de 0.4V la mínima, 0.6V la típica y 0.8V la máxima.
    \item \textbf{IH} (Corriente de mantenimiento): Es la corriente mínima que debe fluir a través del SCR para mantenerlo en estado de conducción después de haber sido disparado. En este caso, es de 0.19mA la típica y 3mA la máxima.
    \item \textbf{tgt} (Tiempo de disparo): Es el tiempo que tarda el SCR en cambiar de estado de bloqueo a estado de conducción después de que se aplica la corriente de disparo a la compuerta. En este caso, es de 1.2µs.
    \item \textbf{tq} (Tiempo de apagado): Es el tiempo que tarda el SCR en volver al estado de bloqueo después de que la corriente a través de él cae por debajo de la corriente de mantenimiento. En este caso, es de 40µs.
    \item \textbf{R$\theta$JC} (Resistencia térmica unión a carcasa): Es la resistencia térmica entre la unión del SCR y su carcasa. En este caso, es de 3°C/W.
    \item \textbf{R$\theta$JA} (Resistencia térmica unión a ambiente): Es la resistencia térmica entre la unión del SCR y el ambiente. En este caso, es de 75°C/W.
\end{itemize}

\subsection{Parametros del TRIAC}

\begin{itemize}
    \item \textbf{VDRM} (Voltaje máximo directo de bloqueo): Es el voltaje máximo que el TRIAC puede soportar en polarización directa sin dispararse. En este caso, es de 600V.
    \item \textbf{IT(RMS)} (Corriente RMS máxima): Es la corriente máxima que el TRIAC puede manejar en condiciones normales de operación. En este caso, es de 8A.
    \item \textbf{ITSM} (Corriente pico de estado de conducción de sobretensión no repetitiva) : Es la corriente máxima que el TRIAC puede manejar en un pulso corto de sobretensión. En este caso, es de 60A.
    \item \textbf{IGT} (Corriente de disparo): Es la corriente mínima que se debe aplicar a la compuerta para disparar el TRIAC. En este caso, es de 10mA.
    \item \textbf{VGT} (Voltaje de disparo): Es el voltaje mínimo que se debe aplicar a la compuerta para disparar el TRIAC. En este caso, es de 1.5V.
    \item \textbf{IH} (Corriente de mantenimiento): Es la corriente mínima que debe fluir a través del TRIAC para mantenerlo en estado de conducción después de haber sido disparado. En este caso, es de 20mA.
    \item \textbf{VTM} (Caída de voltaje en estado de conducción): Es el voltaje que cae a través del TRIAC cuando está en estado de conducción. En este caso, es de 1.3V el mínimo y 1.65V el máximo.
    \item \textbf{tgt} (Tiempo de disparo): Es el tiempo que tarda el TRIAC en cambiar de estado de bloqueo a estado de conducción después de que se aplica la corriente de disparo a la compuerta. En este caso, es de 2µs.
    \item \textbf{R$\theta$j-mb} (Resistencia térmica unión a base de montado): Es la resistencia térmica entre la unión del TRIAC y la base de montado. En este caso, es de 2°C/W.
    \item \textbf{R$\theta$j-a} (Resistencia térmica unión a ambiente): Es la resistencia térmica entre la unión del TRIAC y el ambiente. En este caso, es de 60°C/W.
    \item \textbf{TSTG} (Temperatura de almacenamiento): Es el rango de temperatura en el cual el TRIAC puede ser almacenado sin sufrir daños. En este caso, es de -40°C a 150°C.
    \item \textbf{TJ} (Temperatura de unión): Es el rango de temperatura en el cual el TRIAC puede operar sin sufrir daños. En este caso, es de 125°C.
\end{itemize}

\subsection{Datos no encontrados}
Para TRIAC:
\begin{itemize}
    \item \textbf{tq} (Tiempo de apagado): Es el tiempo que tarda el TRIAC en volver al estado de bloqueo después de que la corriente a través de él cae por debajo de la corriente de mantenimiento. No se encontró dicho dato en el datasheet.
\end{itemize}
Para DIAC:
\begin{itemize}
    \item \textbf{IC} (Desconocido): No se encontró dicho dato en el datasheet.
\end{itemize}
