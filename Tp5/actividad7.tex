\newpage

\section{Actividad 7: Interpretación del Datasheet}

\subsection{Parametros del DIAC}

\subsection{Parametros del SCR}

\begin{itemize}
    \item \textbf{VDRM} (Voltaje máximo directo de bloqueo): Es el voltaje máximo que el SCR puede soportar en polarización directa sin dispararse. En este caso, es de 600V.
    \item \textbf{IT(RMS)} (Corriente RMS máxima): Es la corriente máxima que el SCR puede manejar en condiciones normales de operación. En este caso, es de 4A.
    \item \textbf{IT(AV)} (Corriente media máxima): Es la corriente media máxima que el SCR puede manejar en un ciclo completo. En este caso, es de 2.55A.
    \item \textbf{ITSM} (Corriente de pulso máxima): Es la corriente máxima que el SCR puede manejar en un pulso corto. En este caso, es de 20A.
    \item \textbf{IDRM} (Corriente de fuga en bloqueo directo): Es la corriente que fluye a través del SCR cuando está en polarización directa pero no está disparado. En este caso, es de 10µA.
    \item \textbf{IGT} (Corriente de disparo): Es la corriente mínima que se debe aplicar a la compuerta para disparar el SCR. En este caso, es de 15µA la típica y 200µA la máxima.
    \item \textbf{VGT} (Voltaje de disparo): Es el voltaje mínimo que se debe aplicar a la compuerta para disparar el SCR. En este caso, es de 0.4V la mínima, 0.6V la típica y 0.8V la máxima.
    \item \textbf{IH} (Corriente de mantenimiento): Es la corriente mínima que debe fluir a través del SCR para mantenerlo en estado de conducción después de haber sido disparado. En este caso, es de 0.19mA la típica y 3mA la máxima.
    \item \textbf{tgt} (Tiempo de disparo): Es el tiempo que tarda el SCR en cambiar de estado de bloqueo a estado de conducción después de que se aplica la corriente de disparo a la compuerta. En este caso, es de 1.2µs.
    \item \textbf{tq} (Tiempo de apagado): Es el tiempo que tarda el SCR en volver al estado de bloqueo después de que la corriente a través de él cae por debajo de la corriente de mantenimiento. En este caso, es de 40µs.
    \item \textbf{R$\theta$JC} (Resistencia térmica unión a carcasa): Es la resistencia térmica entre la unión del SCR y su carcasa. En este caso, es de 3°C/W.
    \item \textbf{R$\theta$JA} (Resistencia térmica unión a ambiente): Es la resistencia térmica entre la unión del SCR y el ambiente. En este caso, es de 75°C/W.
\end{itemize}

\subsection{Parametros del TRIAC}