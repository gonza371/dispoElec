\newpage

\section{Actividad 3: Circuitos recortadores con diodos zener}

\section*{Objetivo}
El objetivo de este trabajo es comprender y analizar el uso de los diodos zener en aplicaciones de corriente alterna (CA), específicamente como recortadores de voltaje, mediante la simulación y la implementación práctica de los circuitos.


\subsection{Simulación}
Se simuló el comportamiento de los circuitos a) y b) mostrados en la Figura 4. Estos circuitos utilizan diodos zener para limitar la señal de salida a determinados niveles de tensión.

% AQUÍ VA LA IMAGEN DE LOS CIRCUITOS a) Y b)
\begin{figure}[H]
    \centering
    %\includegraphics[width=0.8\textwidth]{ruta/circuitos_zener.png}
    \caption{Circuitos con diodos zener para recorte de tensión: a) recorte simétrico, b) recorte asimétrico.}
\end{figure}

\subsubsection*{Configuración en LTspice}
\begin{itemize}
    \item Se utilizó una fuente de 24 VCA como entrada.
    \item Se colocó una resistencia limitadora de 1 k$\Omega$ en serie.
    \item Se usaron diodos zener de:
    \begin{itemize}
        \item 5,1 V y 20 V para el circuito a).
        \item 6,8 V y 12 V para el circuito b).
    \end{itemize}
\end{itemize}

\subsubsection*{Resultados de la Simulación}
A continuación se presentan las señales de salida obtenidas para ambos circuitos simulados.

% AQUÍ VA LA IMAGEN DE LA SALIDA DEL CIRCUITO A)
\begin{figure}[H]
    \centering
    %\includegraphics[width=0.7\textwidth]{ruta/salida_circuito_a.png}
    \caption{Señal de salida del circuito a) con recorte simétrico.}
\end{figure}

% AQUÍ VA LA IMAGEN DE LA SALIDA DEL CIRCUITO B)
\begin{figure}[H]
    \centering
    %\includegraphics[width=0.7\textwidth]{ruta/salida_circuito_b.png}
    \caption{Señal de salida del circuito b) con recorte asimétrico.}
\end{figure}

\subsubsection*{Análisis}
\begin{itemize}
    \item En el circuito a), los diodos zener actúan en ambas mitades del ciclo de la señal AC, recortando los picos a aproximadamente ±5,1 V y ±20 V.
    \item En el circuito b), el recorte se realiza de forma asimétrica: uno de los semiperíodos se limita a ±6,8 V y el otro a ±12 V.
\end{itemize}

\subsection{Actividad de Laboratorio}

El objetivo de la actividad es analizar la aplicación de los diodos zener como recortadores. Para ello se implementarán los circuitos a) y b) de la Figura 4.

\textbf{Instrumental y componentes:}
\begin{enumerate}
    \item Osciloscopio
    \item Diodos Zener de 5,1 V, 6,8 V, 12 V y 20 V (todos 2W o 5W según disponibilidad)
    \item Resistencia 1 k$\Omega$ (2W)
    \item Transformador 220 VCA / 24 VCA
\end{enumerate}

\subsection*{Procedimiento}

\textbf{Paso 1: Cálculo de la resistencia limitadora de corriente}

Para limitar la corriente que circula por los diodos zener, se utiliza una resistencia en serie. El valor de esta resistencia se calcula mediante la siguiente fórmula:

\[
R = \frac{V_{\text{in(max)}} - V_Z}{I_Z + I_L}
\]

Donde:
\begin{itemize}
    \item \( V_{\text{in(max)}} \): Tensión máxima de entrada (24 VCA $\rightarrow$ $V_{pico} \approx 33.9$ V)
    \item \( V_Z \): Tensión zener del diodo correspondiente
    \item \( I_Z \): Corriente mínima de conducción del zener (se asume 5 mA)
    \item \( I_L \): Corriente de carga (se asume despreciable)
\end{itemize}

\textbf{Circuito a):}  
Zener de 5,1 V y 20 V conectados en oposición. Se usa el mayor valor (20 V):

\[
R_a = \frac{33.9 - 20}{0.005} = 2780\ \Omega
\]

\textbf{Circuito b):}  
Zener de 6,8 V y 12 V conectados en oposición. Se usa el mayor valor (12 V):

\[
R_b = \frac{33.9 - 12}{0.005} = 4380\ \Omega
\]

En ambos casos se eligió una resistencia de 1 k$\Omega$ (2W), ya que permite el funcionamiento del zener sin sobrepasar los valores de corriente seguros.

\textbf{Paso 2: Armado del circuito a)}

Se armó el circuito a) de la Figura 4 en protoboard, utilizando los componentes indicados.

% AQUÍ VA LA FOTO DEL CIRCUITO A) ARMADO
\begin{figure}[H]
    \centering
    %\includegraphics[width=0.7\textwidth]{ruta/circuito_a_fisico.png}
    \caption{Circuito a) armado en protoboard.}
\end{figure}

\textbf{Paso 3: Medición de la señal de salida del circuito a)}

Se conectó el osciloscopio a la salida y se registró la señal recortada por los zener.

% AQUÍ VA LA CAPTURA DE PANTALLA DEL OSCILOSCOPIO - CIRCUITO A)
\begin{figure}[H]
    \centering
    %\includegraphics[width=0.7\textwidth]{ruta/osciloscopio_circuito_a.png}
    \caption{Señal de salida del circuito a) medida con osciloscopio.}
\end{figure}

\textbf{Paso 4: Armado del circuito b)}

Se armó el circuito b) con los zener de 6,8 V y 12 V, usando también la resistencia de 1 k$\Omega$.

% AQUÍ VA LA FOTO DEL CIRCUITO B) ARMADO
\begin{figure}[H]
    \centering
    %\includegraphics[width=0.7\textwidth]{ruta/circuito_b_fisico.png}
    \caption{Circuito b) armado en protoboard.}
\end{figure}

\textbf{Paso 5: Medición de la señal de salida del circuito b)}

Se registró la forma de onda a la salida del circuito con el osciloscopio.

% AQUÍ VA LA CAPTURA DE PANTALLA DEL OSCILOSCOPIO - CIRCUITO B)
\begin{figure}[H]
    \centering
    %\includegraphics[width=0.7\textwidth]{ruta/osciloscopio_circuito_b.png}
    \caption{Señal de salida del circuito b) medida con osciloscopio.}
\end{figure}

\textbf{Paso 6: Comparación con la simulación}

Se compararon las señales reales obtenidas con las formas de onda obtenidas en LTspice. Se verificó buena concordancia entre los valores de tensión de recorte esperados y medidos.

\textbf{Paso 7: Conclusiones intermedias}

\begin{itemize}
    \item En el circuito a), el recorte es simétrico (±5,1 V y ±20 V).
    \item En el circuito b), el recorte es asimétrico (±6,8 V y ±12 V).
    \item El uso de diodos zener en oposición permite limitar la señal en ambas mitades del ciclo.
    \item La resistencia limitadora es fundamental para proteger los zener y controlar la corriente.
\end{itemize}
\section{Análisis sobre parámetros de hoja de datos}